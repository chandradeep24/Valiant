\documentclass[10pt]{extarticle}
\usepackage[utf8]{inputenc}
\usepackage{mathtools}
\usepackage{amssymb}
\usepackage{listings}
\usepackage{afterpage}
\usepackage{amsthm}
\usepackage[margin = 0.5in]{geometry}
\usepackage{fancyhdr}
\usepackage{framed}
\usepackage{enumitem}
\usepackage{stmaryrd}
\usepackage{multirow}
\newcommand{\N}{\mathbb{N}}
\newcommand{\Z}{\mathbb{Z}}
\newcommand{\Q}{\mathbb{Q}}
\newcommand{\R}{\mathbb{R}}
\newcommand{\C}{\mathbb{C}}
\newcommand{\E}{\mathbb{E}}
\newcommand{\e}{\epsilon}
\renewcommand\labelitemi{-}
\newcommand*\interior[1]{\mathring{#1}}
\newcommand*\closure[1]{\overline{#1}}
\DeclareMathOperator{\im}{Im}
\newtheorem{theorem}{Theorem}
\newtheorem{lemma}[theorem]{Lemma}
\newtheorem{proposition}[theorem]{Proposition}
\newtheorem*{claim}{Claim}
\newtheorem{corollary}[theorem]{Corollary}
\theoremstyle{definition}
\newtheorem{definition}[theorem]{Definition}
\newtheorem{example}[theorem]{Example}


\title{\textbf{Expected Subgraph Capacity for General Graphs}}
\author{Chandradeep Chowdhury \\ cchowdhu@calpoly.edu}
\date{\today}

\begin{document}
% \maketitle

\section{Results}

\begin{definition}
    (\textbf{Interference}) Given two sets $U, W$, we say $U$ $k-$\textit{interferes} with $W$ if 
    \begin{equation}
        |U \cap W| \ge  \frac{|W|}{k}
    \end{equation}
    for some $k \in (0,|W|]$
\end{definition}

\begin{corollary}
    If $|U| = |W|$, then $U$ $k$-interferes with $W$ if and only if $W$ $k$-interferes with $U$.
\end{corollary}

We restrict the upper range of $k$ to $|W|$ for convenience, as beyond that all values of $\frac{|W|}{k}$ will be less than 1.

\begin{definition} (\textbf{(r,T,k)-Subgraph capacity}) Given a vertex set $V = \{v_1,...,v_n\}$, the \textit{$(r,T,k)$-subgraph capacity} of $V$ is the expected maximum number of subgraphs of expected size $r$ that cna be collected subject to the constraint that for any randomly pick subgraph $U$, 
\begin{equation}
    \E[X] \le T
\end{equation}
where $X$ is the number of interferences caused due to $U$. 
\end{definition}

First, we consider the case where all subgraphs are of the same size.

\begin{lemma}
    Given a vertex set $V$ with $n$ vertices and two subsets $U,W$ of size $r$, the probability that $U$ $k$-interferes with $W$ is 
    \begin{equation*}
        \sum_{y = \left\lceil \frac{r}{k} \right\rceil}^{r}  \frac{\binom{r}{y} \binom{n-r}{r-y}}{\binom{n}{r}}
    \end{equation*}
\end{lemma}
\begin{proof}
    if $V = \{v_1,...,v_n\}$, we can represent a subset $U$ as a vector of length $n$, $u$ defined by
    $$
    u_i = \begin{cases}
        1 & \text{if } v_i \in U \\
        0 & \text{if } v_i \notin U
    \end{cases}
    $$
    With this representation, $U,W$ intersect at the indices where both vectors $u, w$ have a $1$. Let $Y$ be a random variable denoting the number of indices where both $u, w$ have a 1. Then 
    \begin{equation}
        \mathbb{P}(Y=y) = \frac{\binom{r}{y} \binom{n-r}{r-y}}{\binom{n}{r}}
    \end{equation}
    This follows from the fact that given the first array $U$, we already know where the 1's are located. We can pick the $y$ intersecting 1's for the second array in $\binom{r}{y}$ ways implicitly placing 0's in the remaining spots. We then fill the remaining $n-r$ indices corresponding to the 0's in the first array with $r-y$ 1's in $\binom{n-r}{r-y}$ ways. Finally we divide by the total number of possible subgraphs $\binom{n}{r}$. 
    Then the probability that $U$ $k$-interferes with $W$ is 
    \begin{equation}
        \sum_{y = \left\lceil \frac{r}{k} \right\rceil}^{r}  \frac{\binom{r}{y} \binom{n-r}{r-y}}{\binom{n}{r}}
    \end{equation}
\end{proof}


\begin{theorem}
    Given a vertex set $V$ with $n$ vertices, the $(r,T,k)$-subgraph capacity of $V$ is 
    \begin{equation*}
        \left\lfloor \frac{T}{\sum_{y = \left\lceil \frac{r}{k} \right\rceil}^{r}  \frac{\binom{r}{y} \binom{n-r}{r-y}}{\binom{n}{r}}} + 1 \right\rfloor
    \end{equation*}
\end{theorem}

\begin{proof}
    Suppose we have $M$ subgraphs in the collection. Pick two subgraphs $U,W$. From lemma 4., we know that the probability of $U$ $k$-interfering with $W$ is $\sum_{y = \left\lceil \frac{r}{k} \right\rceil}^{r}  \frac{\binom{r}{y} \binom{n-r}{r-y}}{\binom{n}{r}}$. Since all subgraphs have the same size, by corollary 2. this is the probability that $U,W$ pair will cause 2 k-interferences. So the expected number of interferences caused by one pair is
    $$
    2\sum_{y = \left\lceil \frac{r}{k} \right\rceil}^{r}  \frac{\binom{r}{y} \binom{n-r}{r-y}}{\binom{n}{r}} 
    $$
    We know that there are $\binom{M}{2} = M(M-1)/2$ such pairings so the expected number of total interferences is
    $$
    2 \frac{M(M-1)}{2} \sum_{y = \left\lceil \frac{r}{k} \right\rceil}^{r}  \frac{\binom{r}{y} \binom{n-r}{r-y}}{\binom{n}{r}}  =  M(M-1) \sum_{y = \left\lceil \frac{r}{k} \right\rceil}^{r}  \frac{\binom{r}{y} \binom{n-r}{r-y}}{\binom{n}{r}} 
    $$
    Since there are $M$ subgraphs, the expected number of interferences by picking one subgraph is
    $$
    \frac{M(M-1)}{M} \sum_{y = \left\lceil \frac{r}{k} \right\rceil}^{r}  \frac{\binom{r}{y} \binom{n-r}{r-y}}{\binom{n}{r}}  = (M-1) \sum_{y = \left\lceil \frac{r}{k} \right\rceil}^{r}  \frac{\binom{r}{y} \binom{n-r}{r-y}}{\binom{n}{r}} 
    $$
    From equation (2), we have 

    \begin{equation*}
        (M-1) \sum_{y = \left\lceil \frac{r}{k} \right\rceil}^{r}  \frac{\binom{r}{y} \binom{n-r}{r-y}}{\binom{n}{r}}  \le T \implies M \le \frac{T}{\sum_{y = \left\lceil \frac{r}{k} \right\rceil}^{r}  \frac{\binom{r}{y} \binom{n-r}{r-y}}{\binom{n}{r}}} + 1
    \end{equation*}
    The $(r,T,k)$-subgraph capacity of $V$ is the largest integer $M$ that satisfies equation (6).
    
\end{proof}

% The following theorem gives an idea of how much the total interference is increased by adding the Mth subgraph. We show that the number is higher than the expected number of interferences caused by adding the subgrap

% \begin{theorem}
%     Given a vertex set $V$ with $n$ vertices and a collection of $M-1$ subgraphs, the expected number of interferences added by adding a random subgraph $U$ with size $r$ is  
%     \begin{equation*}
%         \left\lfloor \frac{T}{\sum_{y = \left\lceil \frac{r}{k} \right\rceil}^{r}  \frac{\binom{r}{y} \binom{n-r}{r-y}}{\binom{n}{r}}} + 1 \right\rfloor
%     \end{equation*}
% \end{theorem}

% \begin{proof}
%     Let $M-1$ be the number of subgraphs of size $r$ currently generated. Add an arbitrary subgraph $U$ of size $r$. From lemma 4., we know that the probability of $U$ $k$-interfering with another subgraph is $\sum_{y = \left\lceil \frac{r}{k} \right\rceil}^{r}  \frac{\binom{r}{y} \binom{n-r}{r-y}}{\binom{n}{r}}$. This can be interpreted as the expected number of additional $k$-interferences $U$ will cause with one subgraph. Since there are $M-1$ other subgraphs, the expected number of $k$-interferences by adding $U$ is 
%     \begin{equation}
%         \sum_{y = \left\lceil \frac{r}{k} \right\rceil}^{r}  \frac{\binom{r}{y} \binom{n-r}{r-y}}{\binom{n}{r}} (M-1)
%     \end{equation}

%     From equation (2), we have 

%     \begin{equation}
%         \sum_{y = \left\lceil \frac{r}{k} \right\rceil}^{r}  \frac{\binom{r}{y} \binom{n-r}{r-y}}{\binom{n}{r}} (M-1) \le T \implies M \le \frac{T}{\sum_{y = \left\lceil \frac{r}{k} \right\rceil}^{r}  \frac{\binom{r}{y} \binom{n-r}{r-y}}{\binom{n}{r}}} + 1
%     \end{equation}
%     The $(r,T,k)$-subgraph capacity of $V$ is the largest integer $M$ that satisfies equation (6).
% \end{proof}

Now we consider the case where the subgraphs might have different sizes but have expected size $r$.

\begin{lemma}
    Given a vertex set $V$ with $n$ vertices and two subsets $U,W$ of size $r_u,r_w$, the probability that $U$ $k$-interferes with $W$ is 
    \begin{equation*}
        \sum_{y = \left\lceil \frac{r_w}{k} \right\rceil}^{r_w}  \frac{\binom{r_u}{y} \binom{n-r_u}{r_w-y}}{\binom{n}{r_w}}
    \end{equation*}
\end{lemma}
\begin{proof}
    We use the same representation as in lemma 4. Note that in this case,
    \begin{equation}
        \mathbb{P}(Y=y) = \frac{\binom{r_u}{y} \binom{n-r_u}{r_w-y}}{\binom{n}{r_w}}
    \end{equation}
    This follows from the fact that given the first array $U$, we already know where the 1's are located. We can pick the $y$ intersecting 1's for the second array in $\binom{r_u}{y}$ ways implicitly placing 0's in the remaining spots. We then fill the remaining $n-r_u$ indices corresponding to the 0's in the first array with $r_w-y$ 1's in $\binom{n-r_u}{r_w-y}$ ways. Finally we divide by the total number of possible subgraphs $\binom{n}{r_w}$. 
    Then the probability that $U$ $k$-interferes with $W$ is 
    \begin{equation}
        \sum_{y = \left\lceil \frac{r_w}{k} \right\rceil}^{r_w}  \frac{\binom{r_u}{y} \binom{n-r_u}{r_w-y}}{\binom{n}{r_w}}
    \end{equation}
\end{proof}

\begin{theorem}
    Given a vertex set $V$ with $n$ vertices, the $(r,T,k)$-subgraph capacity of $V$ is 
    \begin{equation*}
        placeholder
    \end{equation*}
\end{theorem}

\begin{proof}
    Suppose we have $M$ subgraphs $U_1,...,U_m$ with sizes $r_1,...,r_m$. Pick two subgraphs $U_i,U_j$. From lemma 6., we know that the expected number of interferences caused by this pair is
    $$
    \sum_{y = \left\lceil \frac{r_j}{k} \right\rceil}^{r_j}  \frac{\binom{r_i}{y} \binom{n-r_i}{r_j-y}}{\binom{n}{r_j}} + \sum_{y = \left\lceil \frac{r_i}{k} \right\rceil}^{r_i}  \frac{\binom{r_j}{y} \binom{n-r_j}{r_i-y}}{\binom{n}{r_i}}
    $$
    We then sum over all possible pairings to get the expected number of total interferences:
    $$
    \sum_{(i,j) \in (1,m)\times(1,m), i \ne j} \Biggl( \sum_{y = \left\lceil \frac{r_j}{k} \right\rceil}^{r_j}  \frac{\binom{r_i}{y} \binom{n-r_i}{r_j-y}}{\binom{n}{r_j}} + \sum_{y = \left\lceil \frac{r_i}{k} \right\rceil}^{r_i}  \frac{\binom{r_j}{y} \binom{n-r_j}{r_i-y}}{\binom{n}{r_i}} \Biggr)
    $$
    Since there are $M$ subgraphs, the expected number of interferences by picking one subgraph is
    $$
    \frac{1}{M} \sum_{(i,j) \in (1,m)\times(1,m), i \ne j} \Biggl( \sum_{y = \left\lceil \frac{r_j}{k} \right\rceil}^{r_j}  \frac{\binom{r_i}{y} \binom{n-r_i}{r_j-y}}{\binom{n}{r_j}} + \sum_{y = \left\lceil \frac{r_i}{k} \right\rceil}^{r_i}  \frac{\binom{r_j}{y} \binom{n-r_j}{r_i-y}}{\binom{n}{r_i}} \Biggr)
    $$


    
    TODO: we need to use the fact that the expected value is $r$ to simplify this then plug it into equation (2) to solve for $M$. Maybe helpful:

    $$
    \E \left[ \binom{X}{k} \right] = \sum_{0}^{\infty} \binom{i}{k} \mathbb{P}(X=i)
    $$
    
    
    We hope the final result will be similar to the result in theorem 5. 

    % \begin{equation*}
    %     (M-1) \sum_{y = \left\lceil \frac{r}{k} \right\rceil}^{r}  \frac{\binom{r}{y} \binom{n-r}{r-y}}{\binom{n}{r}}  \le T \implies M \le \frac{T}{\sum_{y = \left\lceil \frac{r}{k} \right\rceil}^{r}  \frac{\binom{r}{y} \binom{n-r}{r-y}}{\binom{n}{r}}} + 1
    % \end{equation*}
    % The $(r,T,k)$-subgraph capacity of $V$ is the largest integer $M$ that satisfies equation (6).
    
\end{proof}



\end{document}