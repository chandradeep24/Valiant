\documentclass[10pt]{extarticle}
\usepackage[utf8]{inputenc}
\usepackage{mathtools}
\usepackage{amssymb}
\usepackage{listings}
\usepackage{afterpage}
\usepackage{amsthm}
\usepackage[margin = 0.5in]{geometry}
\usepackage{fancyhdr}
\usepackage{framed}
\usepackage{enumitem}
\usepackage{stmaryrd}
\usepackage{multirow}
\usepackage{natbib}
\usepackage{tikz}
\usepackage[ruled,vlined]{algorithm2e}

\newcommand{\N}{\mathbb{N}}
\newcommand{\Z}{\mathbb{Z}}
\newcommand{\Q}{\mathbb{Q}}
\newcommand{\R}{\mathbb{R}}
\newcommand{\C}{\mathbb{C}}
\newcommand{\E}{\mathbb{E}}
\newcommand{\e}{\epsilon}
\renewcommand\labelitemi{-}
\newcommand*\interior[1]{\mathring{#1}}
\newcommand*\closure[1]{\overline{#1}}
\DeclareMathOperator{\im}{Im}
\newtheorem{theorem}{Theorem}
\newtheorem{lemma}[theorem]{Lemma}
\newtheorem{proposition}[theorem]{Proposition}
\newtheorem*{claim}{Claim}
\newtheorem{corollary}[theorem]{Corollary}
\theoremstyle{definition}
\newtheorem{definition}[theorem]{Definition}
\newtheorem{example}[theorem]{Example}

\renewcommand{\thealgocf}{}

\title{\textbf{Sequence-JOIN Progress Update}}
\author{Patrick Perrine\\ Eben Sherwood}
\date{\today}

\begin{document}
 \maketitle
 
This is to document our current progress in developing Sequence-JOIN (SJOIN). We have created a new process to better account for the cascading effect that firing one $r$-sized memory can have on a realistic Neuroidal model.

\section*{Current Algorithmic Formulation}

\begin{algorithm}
\DontPrintSemicolon
\KwIn{The Neuroidal model $G$, initial memory $A$, and the sequence length $l$.}
\KwOut{An updated $G$, along with the vector of total, chained memories $L$.}
  \SetKwFunction{SJOIN}{SequenceJOIN}
  \SetKwProg{SJ}{Algorithm}{:}{\KwRet}
  \SJ{\SJOIN{$G = (V, E), A, l$}}{
  \For{all synapses $\{i,j\}$ in $E$}{
     $w_{ij} \gets 0$\;
  }
  $L \gets (A)$\;
  \For{$m = 1, 2, \dots l$}{
	  \For{all neurons $i$ in $L_{m}$}{
	       $f_{i} \gets 1$\;
	    }
	 $B \gets \emptyset$\;
	 $k_{2} \gets 2 \cdot k_{G}$\;
	 \For{$k_{test} = k_{2}, k_{2}-1,k_{2}-2, \dots k_{G}$}{ 
	 	 \For{all synapses $\{i,j\}$ in $L_{m}$}{
	       $w_{ij} \gets \frac{T_{i}}{k_{test}}$\;
	    }
	    \texttt{UpdateNeuroids}($G$)\;
		 $B_{test} \gets \emptyset$\;
	    \For{each neuron $i$ in $V$}{
	      \If{$q_{i} == 2$}{
	       $B_{test} \gets B_{test} \cup i$\;
	     }
	    }
	    \If{$|B_{test}| \sim r_{G}$}{
	    	$B \gets B_{test}$\;
	    	\For{all synapses $\{i,j\}$ in $A$}{
	        \If{$j$ is not in $B$}{
	          $w_{ij} \gets 0$\;
	        } 
	      }
	    \textbf{break}\;  
	    }  
	 }
    \eIf{\upshape \texttt{InterferenceCheck}($L, B$)}{
	   \Return{$G, L$}\;
	 }
	 {
	   $L \gets (L, B)$\;
	 }
 	}
 	\Return{$G, L$}\;
}
\caption{Sequence-JOIN for $l$-sized Sequences\label{SeqJ}}
\end{algorithm}

\end{document}
